\def\bayesTh{
  \begin{theorem}[Teorema di Bayes]
    Dati due eventi, $X$ e $Y$, con $\prob{Y}\neq0$, allora vale l'equazione:
    \begin{align*}
      \bayes{X}{Y}
    \end{align*}
    Dove:
    \begin{itemize}
      \item $\prob{X}{Y}$ è la probabilità condizionata rappresentante la verosimiglianza che l'evento $X$ avvenga dato che è avvenuto $Y$.
      \item $\prob{Y}{X}$ è la probabilità condizionata rappresentante la verosimiglianza che l'evento $Y$ avvenga dato che è avvenuto $X$.
      \item $\prob{X}$ è la probabilità a priori di $X$ ed è detta \textbf{probabilità marginale}.
      \item $\prob{Y}$ è la probabilità a priori di $Y$ e funge da costante di normalizzazione.
    \end{itemize}
  \end{theorem}
}