
\def\approxresistancemos{
	\begin{theorem}[Approssimazione della resistenza del canale di un Mosfet]
	La resistenza del canale di un Mos 

	che lavora in zona \textbf{Ohmmica} si puo' stimare:
	\[R_{eq} \approx \frac{1}{k\rnd{V_{GS} - V_T}}\]
	mentre se lavora in zona \textbf{Satura} si puo' stimare:
	\[R_{eq} \approx \frac{1}{2k\rnd{V_{GS} - V_T}}\]
	\end{theorem}
}

\def\potenzadinamica{
	\begin{definition}[Potenza dinamica dissipata]
	E' la potenza dissipata dalla alimentazione in un ciclo completo dei segnali.
	\[P_{din} = \int_{t_0}^{t_1} V_{alim} I_{alim}(t)dt\]
	e per semplificare i conti usiamo direttamente la corrente media.
	\[P_{din} = V_{alim} <I_{alim}>\]
	ed ed applicando la definizione di corrente $I = \frac{\Delta Q}{T}$
	\[I = \frac{\Delta Q}{T_{mcm}}\]
	dove $T_{mcm}$ e' il minimo comune multiplo tra i periodi dei segnali presenti, cosi da compiere un intero ciclo.
	Poiche' di solito la potenza dinamica viene usata per caricare il condensatore d'uscita delle porte logiche abbiamo che (E ricordando che $Q = C \Delta V$):
	\[I = C\frac{\Delta V}{T_{mcm}}\]
	Quindi in ultima analisi ottengo che:
	\[P_{din} = V_{alim} C\frac{\Delta V}{T_{mcm}}\]
	\end{definition}
}

\def\statinmos{
	\begin{theorem}[Stato di funzionamento di un NMOS]
		Un mos puo' essere in 3 stati di funzionamento: Spento, Zona Ohmmica e Saturazione.
		\begin{center}
		    \begin{circuitikz}
		        \draw(0,0) node[nmos] (mos) {}
		        (mos.gate) node[anchor=east] {G}
		        (mos.drain) node[anchor=south] {D}
		        (mos.source) node[anchor=north] {S};
		        \draw (mos.drain) to[open, v^<=$V_{DS}$] (mos.source);
		        \draw (mos.gate)  to[open, v_<=$V_{GS}$] (mos.source);
		        \draw (mos.gate)  to[open, v^<=$V_{GD}$] (mos.drain);
		    \end{circuitikz}
		\end{center}
	\begin{tabular}{c c | c c}
	$V_{GS} > V_T$ & $V_{GD} > V_T$ & Zona di Funzionamento & Corrente\\
	\hline
	No & No & Spento & 0A\\
	Si & No & Saturazione  & $I_{DS} = K_n \left( V_{GS} - V_t \right)^2$\\
	No & Si & Saturazione  & $I_{DS} = - K_n \left( V_{GS} - V_t \right)^2$\\
	Si & Si & Ohmmica & $I_{DS} = K_n \left[ 2 \left(V_{GS} - V_t \right)V_{DS} - V_{DS}^2 \right]$\\


	\end{tabular}

	\end{theorem}
}
\def\statipmos{
	\begin{theorem}[Stato di funzionamento di un PMOS]
		Un mos puo' essere in 3 stati di funzionamento: Spento, Zona Ohmmica e Saturazione.
		\begin{center}
		    \begin{circuitikz} \draw
		        (0,0) node[pmos] (mos) {}
		        (mos.gate) node[anchor=east] {G}
		        (mos.drain) node[anchor=north] {D}
		        (mos.source) node[anchor=south] {S};
		        \draw (mos.drain) to[open, v_>=$V_{SD}$] (mos.source);
		        \draw (mos.gate)  to[open, v^>=$V_{SG}$] (mos.source);
		        \draw (mos.gate)  to[open, v_>=$V_{DG}$] (mos.drain);
		    \end{circuitikz}
		\end{center}

	\begin{tabular}{c c | c c}
	$V_{SG} > V_T$ & $V_{DG} > V_T$ & Zona di Funzionamento & Corrente\\
	\hline
	No & No & Spento & 0A\\
	Si & No & Saturazione  & $ I_{SD} = K_p \left( V_{SG} - V_t \right)^2$\\
	No & Si & Saturazione  & $ I_{SD} = K_p \left( V_{SG} - V_t \right)^2$\\
	Si & Si & Ohmmica & $I_{SD} = K_p \left[ 2 \left(V_{SG} - V_t \right)V_{SD} - V_{SD}^2 \right]$\\


	\end{tabular}

	\end{theorem}
}