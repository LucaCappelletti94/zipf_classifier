\providecommand{\main}{..}
\documentclass[\main/main.tex]{subfiles}
\begin{document}

\chapter{Introduction}
The Zipf law is an empirical law that refers to the fact that in nature, many data can be approximated by a Zipfian distribution, for example texts, some images\footnote{https://www.dcs.warwick.ac.uk/bmvc2007/proceedings/CD-ROM/papers/paper-288.pdf}, even sounds in spoken languages\footnote{https://journals.plos.org/plosone/article?id=10.1371/journal.pone.0033993}. It is therefore of interest to identify ways to exploit this relatively simple way to convert documents into representative vectors in problems such as classifications.



\end{document}