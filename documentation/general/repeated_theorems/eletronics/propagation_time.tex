\def\propgtime{
	\begin{definition}[Tempo di Propagazione]
	Il tempo di propagazione e' il tempo che ci mette la porta ad eseguire meta' della sua escursione.
	Quindi una porta logica che deve transitare da una certa tensione iniziale $V_i$ ad una tensione finale $V_f$ ed ha capacita' equivalente $C$
	Il tempo di propagazione $t_p$ e' il tempo tale che renda vera questa equazione
	\[ V_c(t_p) = V_i + \frac{\rnd{V_f - V_i}}{2}\]
	Osservando che :
	\[ V_i + \frac{\rnd{V_f - V_i}}{2} =  V_f - \frac{\rnd{V_f - V_i}}{2} \]
	\end{definition}
}

\def\propgtimerc{
	\begin{theorem}[Tempo di Propagazione sapendo la  resistenza vista dal condensatore]
	\[t_p = \tau ln\rnd{2} \simeq 0.69\tau \]
	Dove $t_p$ e' il tempo di propagaizone e $\tau$ e' la costante di tempo del condensatore.


	\textbf{Dim:}
	Dalla definizione di tempo di propagazione e dalla risoluzione della equazione caratteristica del condensatore impongo che:
	\[ V_c(t_p) =  V_f - \frac{\rnd{V_f - V_i}}{2}\]
	\[V_c(t)= V_f - \rnd{V_f - V_i}e^{-\frac{t_p}{\tau}}\]
	Dove $V_f$ e' la tensione a cui si carica/scarica il condensatore, $V_i$ e' la tensione da cui parte il condensatore, $t_p$ e' il tempo di propagaizone e $\tau$ e' la costante di tempo del condensatore.

	Dalle quali si deriva:

	\begin{align*}
	 V_f - \frac{\rnd{V_f - V_i}}{2} &= V_f - \rnd{V_f - V_i}e^{-\frac{t_p}{\tau}}\\
	  - \frac{\rnd{V_f - V_i}}{2}  &= - \rnd{V_f - V_i}e^{-\frac{t_p}{\tau}}\\
	  \frac{\rnd{V_f - V_i}}{2}  &= \rnd{V_f - V_i}e^{-\frac{t_p}{\tau}}\\
	  \frac{1}{2} &= e^{-\frac{t_p}{\tau}}\\
	  ln\rnd{\frac{1}{2}} &= -\frac{t_p}{\tau}\\
	 - \tau ln\rnd{\frac{1}{2}} &= t\\
	t_p &= -\tau ln\rnd{2} \simeq 0.69\tau \\
	\end{align*}
	\end{theorem}
}